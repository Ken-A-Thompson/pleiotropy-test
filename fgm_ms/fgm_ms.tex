\documentclass{article}

% packages
\usepackage{lineno} % for line numbers
\usepackage{cite} % for citations (?)
\usepackage{courier}
\usepackage{subscript}


% \usepackage{natbib} % natbib style for bibliography (not sure?)

% references
\begin{document}


\title{Using interspecific hybridization to test a prediction of Fisher's geometric model}
\author{Ken A. Thompson}

\maketitle


% 
% title: |
%   | \vspace{7cm} \LARGE{Phenotype mismatch is common in first-generation hybrids}
% author: "Ken A. Thompson, Mackenzie J. Urquhart-Cronish, Kenneth Whitney, and Dolph Schluter"
% date: '`r Sys.Date()`'
% output:
%   pdf_document:
%     fig_caption: yes
%   html_document: default
%   word_document: default
% header-includes:
% - \usepackage[font={small,it}]{caption}
% - \usepackage{setspace}\doublespacing
% - \usepackage{lineno}
% - \usepackage{courier}
% - \usepackage{marvosym}
% - \usepackage{fixltx2e}
% - \usepackage{amsmath}
% nocite: |
%   @Haldane1924, @Haldane1927
% csl: evolution.csl
% bibliography: /Users/ken.thompson/Documents/library.bib
% ---
% 
% \linenumbers
% \centering
% \raggedright
% \newpage

% <!-- Before putting on biorxiv,use the template (https://www.overleaf.com/project/5cab8731bc0afa62607601a0). or get osmond's! -->

\section{Abstract}

Fisher's geometric model of adaptation predicts that populations undergoing divergent adaptation for a
given trait do so by fixing alleles that affect that trait but have pleiotropic effects on traits under
stabilizing selection. Compensatory mutations fix within populations to counteract this deleterious
pleiotropy. Although compensatory mutations are fixed within populations, inter-population hybridization
causes them to segregate and generate phenotypic variation. A empirically untested prediction of
Fisher's model is that populations that the amount of phenotypic segregation variance released by
hybridization is positively correlated with the amount of divergent in traits under divergent selection.
I systematically searched the literature for studies that measured phenotypic traits in two parent taxa
and their F\textsubscript{1} and F\textsubscript{2} hybrids in a common environment, and find patterns
consistent with those made by Fisher's model. My results suggest that the genes used during divergent
adaptation are pleiotropic and that potentially deleterious segregation variance accumulates systematically as populations diverge.


\newpage

\section{Introduction}

The paper starts off with a citation of \cite{Thompson2015} wackoy!

\bibliographystyle{ieeetr}
\bibliography{library.bib}

\end{document}